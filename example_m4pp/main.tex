\documentclass[a4paper,10pt]{article}

\clubpenalty=10000
\widowpenalty=10000

\usepackage[T1]{fontenc}
\usepackage[english]{babel}
\usepackage{multirow}
\usepackage{tipa}
\usepackage{ucs}
\usepackage[utf8x,utf8]{inputenc}
\usepackage[                         
bookmarks         = true,         
bookmarksnumbered = true,         
colorlinks        = true,         
]{hyperref}                           
\usepackage{color}
\definecolor{webred}{rgb}{0.3,0,0}
\definecolor{blurl}{rgb}{0,0,0.3}
\hypersetup{
linkcolor         = webred, %black
citecolor         = webred, %black
urlcolor          = blurl , %black
linkbordercolor   = {1 1 1},
citebordercolor   = {1 1 1},
urlbordercolor    = {1 1 1},
pdfauthor   = {},
pdftitle    = {Self Documenting Example},
pdfsubject  = {},
pdfkeywords = {},
pdfcreator  = {Bracetax and PDFLaTeX},
pdfproducer = {Bracetax and PDFLaTeX}}

\usepackage[pdftex]{graphicx}
\frenchspacing
\DeclareGraphicsExtensions{.jpg,.mps,.pdf,.png}

%Generated with BraceTax


\begin{document}

 % HEADER:
\date{}\title{Self-{}documenting authoring example}
\author{Sebastien Mondet}
\date{Using m4\ldots{}}
\maketitle


  
\section{Some links}
\label{seclinks}
  We will use \href{http://en.wikipedia.org/wiki/M4_(computer_language)}{M4}   
\section{Detailled examples}
\label{secexamples}
  
\subsection{Creating a ``ToDo'' tag}
\label{secdeftodo}
    Here is the M4 code: %
%verbatimbegin:m4_code
\begin{verbatim}
define(TODO, ifelse(_output_format, HTML ,
{bypass end}<span style="color: red; background-color: yellow">TO-DO</span>{end},
{bypass end}\textcolor{red}{TO-DO}{end}))dnl
\end{verbatim}
%verbatimend:m4_code
 so, with the command: %
%verbatimbegin:shell_code
\begin{verbatim}
m4  -D_output_format="LATEX"  main.brtx | brtx [-html|-latex]
\end{verbatim}
%verbatimend:shell_code
 with a \texttt{TODO}, the output generated by \texttt{brtx}, will have a \textcolor{red}{TO-DO}    
\subsection{Using command line definitions}
\label{secdatecommand}
  Here we have the compilation date: \texttt{Sun Oct  2 14:40:43 EDT 2011} \par{}
 For that we just passed the source: %
%verbatimbegin:brtx_code
\begin{verbatim}
Here we have compilation date: {t|_todaydate}
\end{verbatim}
%verbatimend:brtx_code
 through the command: %
%verbatimbegin:shell_code
\begin{verbatim}
m4 -D_todaydate="$(date)" main.brtx > main.brtx.tmp
\end{verbatim}
%verbatimend:shell_code
  
\subsection{Citations}
\label{secbiblio}
  Let's say we want to use BibTeX for LaTeX output and that we have a bibliography site (\texttt{.\slash{}\linebreak[0]biblio\slash{}\linebreak[0]<refname>.html} or \texttt{.\slash{}\linebreak[0]biblio.html\#{}<refname>}).   We define CITATION: %
\begin{verbatim}
define(CITATION, ifelse(_output_format , HTML,
[{link ./biblio/$1.html|$1}],
{bypass end}\cite{$1}{end}))dnl
\end{verbatim} so that %
\begin{verbatim}
CITATION(mondet08bracetax)
\end{verbatim} will give us: %
\begin{verbatim}
{bypass end}\cite{mondet08bracetax}{end}
\end{verbatim}   
\subsection{Using \texttt{include}}
\label{secinclude}
  With a simple: %
%verbatimbegin:brtx_code
\begin{verbatim}
{code verycomplexend}
include(./Makefile)
{verycomplexend}
\end{verbatim}
%verbatimend:brtx_code
 we obtain: %
\begin{verbatim}

EXAMPLE_SITE_HTML=main.html main.tex

EXAMPLE_SITE_VIM= Makefile.html main.brtx.html

EXAMPLE_SITE= $(EXAMPLE_SITE_HTML) $(EXAMPLE_SITE_VIM) main.pdf main.tex

all: $(EXAMPLE_SITE)

.PHONY: nopdf

nopdf: $(EXAMPLE_SITE_HTML) $(EXAMPLE_SITE_VIM)

#
# Generate the HTML page from source:
#                         m4              brtx -html       brtx -prospro    
#   source bracetax + M4 ---> pure bracetax  --->    HTML       --->     HTML
#
HEVEA_OR_NOT= hevea || echo NoHeveaFound
main.html: main.brtx
	m4 -D_todaydate="date`" -D_output_format="HTML" main.brtx > main.brtx.tmp
	$(BRTX2HTML) -doc -title "Bracetax Example" -link-css ../$(CSS) \
	    -i main.brtx.tmp -o $@
	rm main.brtx.tmp


#
# Generate the HTML page from source:
#                         m4              brtx -html       brtx -prospro    
#   source bracetax + M4 ---> pure bracetax  --->    HTML       --->     HTML
#
main.tex: main.brtx
	m4 -D_todaydate="`date`" -D_output_format="LATEX" main.brtx > main.brtx.tmp
	brtx -doc -title "Self Documenting Example" -latex  -i main.brtx.tmp -o main.tex
	rm main.brtx.tmp 

#
# A PDF with pdflatex
#
main.pdf: main.tex
	pdflatex main && pdflatex main
	rm main.log main.aux main.out

#
# Using VIM to generate an HTML page from a source code
#
Makefile.html: Makefile
	vim -f -R $< \
		-c "sy on" \
		-c 'colorscheme darkblue' \
		-c TOhtml -c 'w! $@.tmp -c 'qa!' && \
	sed -e 's/<title>.*<\/title>/<title>Makefile<\/title>/' \
		$@.tmp > $@
	rm -f $@.tmp

#
# Using VIM to generate an HTML page from a source code
# (with the provided - incomplete - vim plugin
#
main.brtx.html: main.brtx
	vim -f -R $< \
		-c "sy on" \
		-c 'colorscheme darkblue' \
		-c 'source ../../../tools/bracetax_syntax.vim' \
		-c TOhtml -c 'w! $@.tmp' -c 'qa!' && \
	sed -e 's/<title>.*<\/title>/<title>Example - source<\/title>/' \
		$@.tmp > $@ 
	rm -f $@.tmp


\end{verbatim}   
\end{document}
